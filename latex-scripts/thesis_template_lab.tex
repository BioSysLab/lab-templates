\documentclass[10pt,a4paper, openany]{book}
\usepackage{amsmath}	%packages for writing equations
\usepackage{amsfonts}
\usepackage{amssymb}

% packages to incorporate greek writing
\usepackage[utf8]{inputenc}		%encoding of the input
\usepackage[greek,english]{babel} %babel package to include greek hyphens
\usepackage[LGR, T1]{fontenc}	%encoding of the font
\usepackage{babelbib}	%babel package for bibliography
\usepackage{csquotes}	%needed package for bibliography
%\usepackage{kmath, kerkis}	%installing kerkis package. 
							%YOU HAVE TO FIRST INSTALL KERKIS ON COMPUTER!

\usepackage{multirow}	

\usepackage{alphabeta}
\usepackage{csvsimple}
\usepackage{booktabs}
\usepackage[section]{placeins}


% using hyperlinks for table of contents etc.
\usepackage[unicode]{hyperref}
\hypersetup{
	linktoc = all,		%making table of contents clickable
	linktocpage = false	%table of contents linked to paragraph, not page
}

% formatting first page
\usepackage[tc]{titlepic}
\usepackage{xcolor}
\usepackage{graphicx}
\usepackage{rotating}
\usepackage{caption}
\usepackage{subcaption}
\usepackage[executivepaper,margin=1in]{geometry}

\usepackage{longtable}

\definecolor{black}{RGB}{0,0,0}
\graphicspath{{C:/Users/Konstantinos/Desktop/thesis/}}	%image directory

% Configuring header and footer
\usepackage{fancyhdr}
\fancyhf{}
\fancyheadoffset{0cm}
\renewcommand{\headrulewidth}{0pt}
\renewcommand{\footrulewidth}{0pt}
\fancyhead[R]{\thepage}
\fancypagestyle{plain}{%
	\fancyhf{}%
	\fancyhead[R]{\thepage}%
}

\usepackage{blindtext}
\usepackage{parskip}


\DeclareMathOperator{\proj}{proj}
\DeclareMathOperator{\cov}{cov}
 
\begin{document}	
\latintext	

\begin{titlepage}
	\centering
	\vspace*{1cm}
	\includegraphics[width=0.4\textwidth]{logo_ntua.jpg}\\[1cm]
	University of Patras \\[5pt]
	Science Department \\[5pt]
	Theoretical, Computational Physics and Astrophysics\\[1cm]
	{\Large\scshape Research Paper \par}
	{\huge\bfseries Title \par}




	\thispagestyle{empty}
\end{titlepage}

\setcounter{page}{1}
\frontmatter	%numbering is latin numbers, this is the intro

\chapter*{Ευχαριστίες} %asterisk to prevent chapter from being numbered
Θα ήθελα να ευχαριστήσω θερμά	\ldots .

\chapter*{Περίληψη}
Η μοντελοποίηση βιολογικών συστημάτων είναι ένα από τα πολλά παρακλάδια του ταχύτατα αναπτυσσόμενου κλάδου της υπολογιστικής βιολογίας. 

\vfill  %transfer next paragraph to end of page

%two commands to change to english from greek. textlatin{} contains english text between the braces, latintext turns all text following to english and needs command greektext to turn to greek again. RECOMMENDED: use textlatin with braces when writing a dissertation in greek.
Λέξεις κλειδιά: υπολογιστική βιολογία, καρκίνος, μελάνωμα, βιολογικά μονοπάτια, σύνθετα συστήματα, διαφορικώς εκφρασμένα γονίδια,βιοπληροφορική, support vector machines, επανατοποθέτηση φαρμάκων,\latintext machine learning, cMap.



\chapter*{Abstract}
Systems biology is one of the many branches of the rapidly evolving science of computational biology. 
\vfill
Key words: computational biology, cancer, melanoma, biological pathways, complex systems, differentially expressed genes, bioinformatics, drug, support vector machines, repurposing, machine learning, cMap.\selectlanguage{greek}

\tableofcontents	%adds table of contents
\mainmatter		%numbering turns to numbers
\chapter{Εισαγωγή}
\label{ch:intro}	%use labels to reference in the future
					%structures are chapter->section->subsection
\section{Υπολογιστική Βιολογία}
\label{sec:compbio}
Η υπολογιστική βιολογία είναι ένας ευρύς επιστημονικός κλάδος που συνδυάζει γνώσεις από πολλές επιστήμες \cite{lykeio}. 



% add bibliography
%\bibliographystyle{babunsrt}
%\bibliography{yourbib} 
\end{document}
